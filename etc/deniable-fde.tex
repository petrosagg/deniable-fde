\documentclass{acm_proc_article-sp}

\begin{document}

\title{A provably deniable full-disk encryption system}

\numberofauthors{2} 
\author{
\alignauthor
Petros Angelatos\\
       \affaddr{resin.io}\\
       \email{petrosagg@resin.io}
\alignauthor
Dionysis Zindros\\
       \affaddr{University of Athens}\\
       \email{dionyziz@gmail.com}
}

\date{June 2016}

\maketitle
\begin{abstract}
Deniable full-disk encryption has been a long-standing practical problem. While
there have been numerous attempts at attacking the problem, they either do not
achieve full indistinguishability or are impractical to implement while
maintaining backwards compatibility with existing mainstream operating systems.
In this work, we introduce a formal model for the successful
indistinguishability requirements of deniable full-disk encryption. We present
a deniable full-disk encryption scheme based on LVM thin provisioning which
works in a backwards-compatible manner on existing popular Linux distributions
such as Ubuntu, prove its indistinguishability, and provide a complete
production ready implementation.
\end{abstract}

\section{Previous work}

We are exploring the possibility of constructing a deniable FDE scheme based on
the premise of indistinguishability from popular operating system
installations.

Our deniable FDE implementation is based on extending the MobiPluto paper
\cite{c0} and our model is based on existing desirable features for deniable
FDE \cite{c1}.

Existing solutions for deniable FDE are insufficient. Solutions that offer
deniability as their main feature, such as MobiPluto, are obviously
self-invalidating, as their very use brings suspicion of deniability use. Other
existing solutions are either no longer maintained, such as TrueCrypt \cite{c5}
developed by anonymous hackers and Rubberhose \cite{c6} developed by Julian
Assange, Suelette Dreyfus, etc., or purely theoretical and without an
implementation.

\section{Our contributions}

Firstly, there seems to be no formal model for describing deniable FDE
features. We introduce a more precise model of deniability in an FDE setting
based on the plausibly deniable encryption paper, which considers some informal
characteristics and features.

Additionally, we provide a practical production-grade implementation for the
Linux kernel which achieves indistinguishability against the Ubuntu system.

\section{Formal full-disk deniability}

In our description, we compare two world views that the deniability adversary
is comparing: First, the view of a hard drive with deniability disabled
(Alice); and second, the view of a hard drive with deniability enabled (Bob).

In the threat model, the adversary can take a single hard drive snapshot only
once during their use, i.e. the attacker does not have multiple snapshots
during different times of hard drive use. This is following Schneier's
"one-time access" restriction \cite{c4}. Then, we say that FDE satisfies the
deniability property if the snapshot of Alice's hard drive along with the FDE
decryption password and Bob's hard drive providing the decoy FDE password are
informationally indistinguishable. Specifically, for all Bob's hard drive
snapshots accompanied with an FDE decoy decryption key, there exists an Alice's
hard drive snapshot accompanied with the same FDE decryption key which is
identical.

Snapshots of hard drive data are treated as simple strings – no additional
physical information such as physical hard drive state is considered. This is
important to note, as forensic analysis of physical properties could destroy
deniability. Note also that this theoretical definition may not be enough for a
practical defence: The decoy system has to remain in occasional use in order to
be plausible; a system that remains unused for years is suspicious. At least in
the full-disk deniable encryption case, we can provide some strong assurances
that the data is indistinguishable a the block level (in cases where a part of
the disk drive is deniably encrypted, and not a whole operating system,
deniable disk encryption fails due to metadata \cite{c3}).

Any proofs of security need to assume that AES or in general the symmetric
cipher used to achieve disk encryption behaves like a random permutation.

\section{The Ubuntu decoy system}

The view of Alice must be a hard drive view which is constructed by popular
operating systems during installation without unusual configuration. In our
construction, we choose as the view of Alice the default Ubuntu installation in
which full-disk encryption is enabled, LVM is enabled [see Figure 1 – both
checkmarks must be ticked], and the user has chosen to wipe the disk with
randomness [see Figure 2 – the checkmark must be ticked]. This is a plausible
system, because many people are using such configurations, as it is as simple
as ticking two checkboxes during Ubuntu's installation (statistical
experimental data on how popular this installation scheme is should be
collected for completeness). Similar options also exist for alternative
distributions such as Mint, but in this first attempt we will concentrate on
indistinguishability from Ubuntu. Then Alice's view is shown in Figure 3.

We assume that one physical disk is used and that three partitions are created
during the installation: One boot partition, one swap partition, and one root
partition. Under these assumptions, the view of the hard drive based on the
default installation parameters is as above. The MBR section at the beginning
of the disk drive is 512 bytes and contains unencrypted code that runs during
boot and jumps to the /boot partition for loading the boot loader. Space A is a
small unallocated block which contains uniformly random data and is 1MB - 512
in size (i.e. |MBR| + |Space A| = 1MB). Following is the /boot partition, which
contains the unencrypted boot loader and kernel. Following the /boot partition
is Space B, which is another unallocated disk block 1MB in size. After that,
the LVM partition appears, which is used as the physical volume storing the LVM
volume group, and takes up the rest of the disk drive. The LVM volume group
contains two logical volumes, the /root partition and the swap partition. The
LVM physical volume of which the LVM volume group is constructed is a linux
block device and is encrypted at the block level using LUKS.

Due to LVM being in use in Alice's case, blocks will be allocated at
partitioning time to construct a mapping between the physical LVM volume and
the logical volume blocks. This mapping information will be stored unencrypted
in a special location at the beginning of the LVM physical volume \cite{c7}.
When data is written to the / or the swap logical partitions, they will be
mapped by LVM to the appropriate physical volume block, which will then be
written to disk.

The blocks of the physical volume that correspond to logical volume blocks that
have never been written to will contain uniformly random data due to the
original wiping applied during OS installation.

\section{Our deniable file system}

In our proposal, we allow Bob to have a hard drive which is indistinguishable
from Alice’s hard drive. However, a small additional USB drive is required.
This USB stick contains only data that is only available. It does not contain
secret or encrypted data. The USB stick is a modification of the standard
Ubuntu LiveUSB. It contains the appropriate boot loader so that a deniable
operating system can be booted and the deniable partition can be mounted, given
the appropriate passwords. The USB stick also allows Bob to boot into the decoy
system while mounting it safely. It is critical that this USB stick is not
found by the adversary; the existence of such a USB stick would hint at the
desire for deniability and can persuade the adversary to inquire further.
However, the USB stick can be destroyed at any time without loss of any data,
as it can be downloaded again in the future.

The deniable operating system installation process begins by Bob booting from
the deniable USB stick. Bob then inputs a deniable full-disk encryption
password. That password is different from the full-disk encryption password
used by the decoy operating system, which we call the decoy password.
Initially, the installation proceeds to fully wipe the target hard drive with
uniformly random data.

The deniable installer first creates an LVM thin-provisioning block device.

There are three ways to mount this system:

Safely mount decoy operating system

Mount deniable operating system

Mount decoy operating system

TODO: Explain them in detail. How are these mounted?

When the decoy operating system operates safely, it can write to the disk
freely. The following cases exist:

An unallocated block is being written to

An allocated block is being written to

The metadata block is being written to

TODO: Explain the cases and ensure each of them is “safe” (i.e. retains
indistinguishability through (1) inaction, (2) our block reshuffling technique,
(3) our metadata reallocation technique)

We assume that the LVM physical volume will have some blocks allocated to it
which remain unused, i.e. they have never been written to.

We require some data to be stored on a USB stick that will be used for booting.
This USB stick leaves no trace on the machine it's plugged into, a requirement
put forth by Schneier \cite{c7}. This USB stick contains no secret data, but it
is revealing of the fact that distinguishability is in use. It provides the
boot loader for Bob’s system.

\section{TO DO}
How do we deal with SSD TRIM operations and random wiping?

How to regain space lost due to deleted files?

How to safely mount decoy system without leaking metadata?

\section{A proof of deniability}

TODO

\section{Future work}

Changes of major distros that will allow using deniable FDE without external
boot device

Bootloader integration

\begin{thebibliography}{13}

\bibitem{c1} Chang, Bing, Zhan Wang, Bo Chen, and Fengwei Zhang. "MobiPluto: File System Friendly Deniable Storage for Mobile Devices." In Proceedings of the 31st Annual Computer Security Applications Conference, pp. 381-390. ACM, 2015.

\bibitem{c2} Björgvin Ragnarsson, Gabor Toth, Hanieh Bagheri, and Wicher Minnaard. "Desirable features for plausibly deniable encryption." (2012).

\bibitem{c3} Czeskis, Alexei, David J. St Hilaire, Karl Koscher, Steven D. Gribble, Tadayoshi Kohno, and Bruce Schneier. "Defeating Encrypted and Deniable File Systems: TrueCrypt v5. 1a and the Case of the Tattling OS and Applications." In HotSec. 2008.

\bibitem{c4} [online] URL: \url{https://www.schneier.com/blog/archives/2008/07/truecrypts_deni.html} [cited May 2016]

\bibitem{c5} [online] URL: \url{https://truecrypt.ch/} [cited May 2016]

\bibitem{c6} [online] URL: \url{https://web.archive.org/web/20110726185300/http://iq.org/~proff/rubberhose.org/} [cited May 2016]

\bibitem{c7} [online] URL: \url{https://www.schneier.com/blog/archives/2006/04/deniable_file_s.html} [cited May 2016]

\end{thebibliography}

%\balancecolumns 

\end{document}
